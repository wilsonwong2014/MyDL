
% Default to the notebook output style

    


% Inherit from the specified cell style.




    
\documentclass[11pt]{article}

    \usepackage{fontspec, xunicode,xltxtra}
    \setmainfont{uming}
    
    \usepackage[T1]{fontenc}
    % Nicer default font (+ math font) than Computer Modern for most use cases
    \usepackage{mathpazo}

    % Basic figure setup, for now with no caption control since it's done
    % automatically by Pandoc (which extracts ![](path) syntax from Markdown).
    \usepackage{graphicx}
    % We will generate all images so they have a width \maxwidth. This means
    % that they will get their normal width if they fit onto the page, but
    % are scaled down if they would overflow the margins.
    \makeatletter
    \def\maxwidth{\ifdim\Gin@nat@width>\linewidth\linewidth
    \else\Gin@nat@width\fi}
    \makeatother
    \let\Oldincludegraphics\includegraphics
    % Set max figure width to be 80% of text width, for now hardcoded.
    \renewcommand{\includegraphics}[1]{\Oldincludegraphics[width=.8\maxwidth]{#1}}
    % Ensure that by default, figures have no caption (until we provide a
    % proper Figure object with a Caption API and a way to capture that
    % in the conversion process - todo).
    \usepackage{caption}
    \DeclareCaptionLabelFormat{nolabel}{}
    \captionsetup{labelformat=nolabel}

    \usepackage{adjustbox} % Used to constrain images to a maximum size 
    \usepackage{xcolor} % Allow colors to be defined
    \usepackage{enumerate} % Needed for markdown enumerations to work
    \usepackage{geometry} % Used to adjust the document margins
    \usepackage{amsmath} % Equations
    \usepackage{amssymb} % Equations
    \usepackage{textcomp} % defines textquotesingle
    % Hack from http://tex.stackexchange.com/a/47451/13684:
    \AtBeginDocument{%
        \def\PYZsq{\textquotesingle}% Upright quotes in Pygmentized code
    }
    \usepackage{upquote} % Upright quotes for verbatim code
    \usepackage{eurosym} % defines \euro
    \usepackage[mathletters]{ucs} % Extended unicode (utf-8) support
    \usepackage[utf8x]{inputenc} % Allow utf-8 characters in the tex document
    \usepackage{fancyvrb} % verbatim replacement that allows latex
    \usepackage{grffile} % extends the file name processing of package graphics 
                         % to support a larger range 
    % The hyperref package gives us a pdf with properly built
    % internal navigation ('pdf bookmarks' for the table of contents,
    % internal cross-reference links, web links for URLs, etc.)
    \usepackage{hyperref}
    \usepackage{longtable} % longtable support required by pandoc >1.10
    \usepackage{booktabs}  % table support for pandoc > 1.12.2
    \usepackage[inline]{enumitem} % IRkernel/repr support (it uses the enumerate* environment)
    \usepackage[normalem]{ulem} % ulem is needed to support strikethroughs (\sout)
                                % normalem makes italics be italics, not underlines
    \usepackage{mathrsfs}
    

    
    
    % Colors for the hyperref package
    \definecolor{urlcolor}{rgb}{0,.145,.698}
    \definecolor{linkcolor}{rgb}{.71,0.21,0.01}
    \definecolor{citecolor}{rgb}{.12,.54,.11}

    % ANSI colors
    \definecolor{ansi-black}{HTML}{3E424D}
    \definecolor{ansi-black-intense}{HTML}{282C36}
    \definecolor{ansi-red}{HTML}{E75C58}
    \definecolor{ansi-red-intense}{HTML}{B22B31}
    \definecolor{ansi-green}{HTML}{00A250}
    \definecolor{ansi-green-intense}{HTML}{007427}
    \definecolor{ansi-yellow}{HTML}{DDB62B}
    \definecolor{ansi-yellow-intense}{HTML}{B27D12}
    \definecolor{ansi-blue}{HTML}{208FFB}
    \definecolor{ansi-blue-intense}{HTML}{0065CA}
    \definecolor{ansi-magenta}{HTML}{D160C4}
    \definecolor{ansi-magenta-intense}{HTML}{A03196}
    \definecolor{ansi-cyan}{HTML}{60C6C8}
    \definecolor{ansi-cyan-intense}{HTML}{258F8F}
    \definecolor{ansi-white}{HTML}{C5C1B4}
    \definecolor{ansi-white-intense}{HTML}{A1A6B2}
    \definecolor{ansi-default-inverse-fg}{HTML}{FFFFFF}
    \definecolor{ansi-default-inverse-bg}{HTML}{000000}

    % commands and environments needed by pandoc snippets
    % extracted from the output of `pandoc -s`
    \providecommand{\tightlist}{%
      \setlength{\itemsep}{0pt}\setlength{\parskip}{0pt}}
    \DefineVerbatimEnvironment{Highlighting}{Verbatim}{commandchars=\\\{\}}
    % Add ',fontsize=\small' for more characters per line
    \newenvironment{Shaded}{}{}
    \newcommand{\KeywordTok}[1]{\textcolor[rgb]{0.00,0.44,0.13}{\textbf{{#1}}}}
    \newcommand{\DataTypeTok}[1]{\textcolor[rgb]{0.56,0.13,0.00}{{#1}}}
    \newcommand{\DecValTok}[1]{\textcolor[rgb]{0.25,0.63,0.44}{{#1}}}
    \newcommand{\BaseNTok}[1]{\textcolor[rgb]{0.25,0.63,0.44}{{#1}}}
    \newcommand{\FloatTok}[1]{\textcolor[rgb]{0.25,0.63,0.44}{{#1}}}
    \newcommand{\CharTok}[1]{\textcolor[rgb]{0.25,0.44,0.63}{{#1}}}
    \newcommand{\StringTok}[1]{\textcolor[rgb]{0.25,0.44,0.63}{{#1}}}
    \newcommand{\CommentTok}[1]{\textcolor[rgb]{0.38,0.63,0.69}{\textit{{#1}}}}
    \newcommand{\OtherTok}[1]{\textcolor[rgb]{0.00,0.44,0.13}{{#1}}}
    \newcommand{\AlertTok}[1]{\textcolor[rgb]{1.00,0.00,0.00}{\textbf{{#1}}}}
    \newcommand{\FunctionTok}[1]{\textcolor[rgb]{0.02,0.16,0.49}{{#1}}}
    \newcommand{\RegionMarkerTok}[1]{{#1}}
    \newcommand{\ErrorTok}[1]{\textcolor[rgb]{1.00,0.00,0.00}{\textbf{{#1}}}}
    \newcommand{\NormalTok}[1]{{#1}}
    
    % Additional commands for more recent versions of Pandoc
    \newcommand{\ConstantTok}[1]{\textcolor[rgb]{0.53,0.00,0.00}{{#1}}}
    \newcommand{\SpecialCharTok}[1]{\textcolor[rgb]{0.25,0.44,0.63}{{#1}}}
    \newcommand{\VerbatimStringTok}[1]{\textcolor[rgb]{0.25,0.44,0.63}{{#1}}}
    \newcommand{\SpecialStringTok}[1]{\textcolor[rgb]{0.73,0.40,0.53}{{#1}}}
    \newcommand{\ImportTok}[1]{{#1}}
    \newcommand{\DocumentationTok}[1]{\textcolor[rgb]{0.73,0.13,0.13}{\textit{{#1}}}}
    \newcommand{\AnnotationTok}[1]{\textcolor[rgb]{0.38,0.63,0.69}{\textbf{\textit{{#1}}}}}
    \newcommand{\CommentVarTok}[1]{\textcolor[rgb]{0.38,0.63,0.69}{\textbf{\textit{{#1}}}}}
    \newcommand{\VariableTok}[1]{\textcolor[rgb]{0.10,0.09,0.49}{{#1}}}
    \newcommand{\ControlFlowTok}[1]{\textcolor[rgb]{0.00,0.44,0.13}{\textbf{{#1}}}}
    \newcommand{\OperatorTok}[1]{\textcolor[rgb]{0.40,0.40,0.40}{{#1}}}
    \newcommand{\BuiltInTok}[1]{{#1}}
    \newcommand{\ExtensionTok}[1]{{#1}}
    \newcommand{\PreprocessorTok}[1]{\textcolor[rgb]{0.74,0.48,0.00}{{#1}}}
    \newcommand{\AttributeTok}[1]{\textcolor[rgb]{0.49,0.56,0.16}{{#1}}}
    \newcommand{\InformationTok}[1]{\textcolor[rgb]{0.38,0.63,0.69}{\textbf{\textit{{#1}}}}}
    \newcommand{\WarningTok}[1]{\textcolor[rgb]{0.38,0.63,0.69}{\textbf{\textit{{#1}}}}}
    
    
    % Define a nice break command that doesn't care if a line doesn't already
    % exist.
    \def\br{\hspace*{\fill} \\* }
    % Math Jax compatibility definitions
    \def\gt{>}
    \def\lt{<}
    \let\Oldtex\TeX
    \let\Oldlatex\LaTeX
    \renewcommand{\TeX}{\textrm{\Oldtex}}
    \renewcommand{\LaTeX}{\textrm{\Oldlatex}}
    % Document parameters
    % Document title
    \title{demo\_bp}
    
    
    
    
    

    % Pygments definitions
    
\makeatletter
\def\PY@reset{\let\PY@it=\relax \let\PY@bf=\relax%
    \let\PY@ul=\relax \let\PY@tc=\relax%
    \let\PY@bc=\relax \let\PY@ff=\relax}
\def\PY@tok#1{\csname PY@tok@#1\endcsname}
\def\PY@toks#1+{\ifx\relax#1\empty\else%
    \PY@tok{#1}\expandafter\PY@toks\fi}
\def\PY@do#1{\PY@bc{\PY@tc{\PY@ul{%
    \PY@it{\PY@bf{\PY@ff{#1}}}}}}}
\def\PY#1#2{\PY@reset\PY@toks#1+\relax+\PY@do{#2}}

\expandafter\def\csname PY@tok@o\endcsname{\def\PY@tc##1{\textcolor[rgb]{0.40,0.40,0.40}{##1}}}
\expandafter\def\csname PY@tok@fm\endcsname{\def\PY@tc##1{\textcolor[rgb]{0.00,0.00,1.00}{##1}}}
\expandafter\def\csname PY@tok@cpf\endcsname{\let\PY@it=\textit\def\PY@tc##1{\textcolor[rgb]{0.25,0.50,0.50}{##1}}}
\expandafter\def\csname PY@tok@bp\endcsname{\def\PY@tc##1{\textcolor[rgb]{0.00,0.50,0.00}{##1}}}
\expandafter\def\csname PY@tok@ss\endcsname{\def\PY@tc##1{\textcolor[rgb]{0.10,0.09,0.49}{##1}}}
\expandafter\def\csname PY@tok@kr\endcsname{\let\PY@bf=\textbf\def\PY@tc##1{\textcolor[rgb]{0.00,0.50,0.00}{##1}}}
\expandafter\def\csname PY@tok@vm\endcsname{\def\PY@tc##1{\textcolor[rgb]{0.10,0.09,0.49}{##1}}}
\expandafter\def\csname PY@tok@kt\endcsname{\def\PY@tc##1{\textcolor[rgb]{0.69,0.00,0.25}{##1}}}
\expandafter\def\csname PY@tok@si\endcsname{\let\PY@bf=\textbf\def\PY@tc##1{\textcolor[rgb]{0.73,0.40,0.53}{##1}}}
\expandafter\def\csname PY@tok@gp\endcsname{\let\PY@bf=\textbf\def\PY@tc##1{\textcolor[rgb]{0.00,0.00,0.50}{##1}}}
\expandafter\def\csname PY@tok@sx\endcsname{\def\PY@tc##1{\textcolor[rgb]{0.00,0.50,0.00}{##1}}}
\expandafter\def\csname PY@tok@ne\endcsname{\let\PY@bf=\textbf\def\PY@tc##1{\textcolor[rgb]{0.82,0.25,0.23}{##1}}}
\expandafter\def\csname PY@tok@sc\endcsname{\def\PY@tc##1{\textcolor[rgb]{0.73,0.13,0.13}{##1}}}
\expandafter\def\csname PY@tok@sd\endcsname{\let\PY@it=\textit\def\PY@tc##1{\textcolor[rgb]{0.73,0.13,0.13}{##1}}}
\expandafter\def\csname PY@tok@err\endcsname{\def\PY@bc##1{\setlength{\fboxsep}{0pt}\fcolorbox[rgb]{1.00,0.00,0.00}{1,1,1}{\strut ##1}}}
\expandafter\def\csname PY@tok@mh\endcsname{\def\PY@tc##1{\textcolor[rgb]{0.40,0.40,0.40}{##1}}}
\expandafter\def\csname PY@tok@nn\endcsname{\let\PY@bf=\textbf\def\PY@tc##1{\textcolor[rgb]{0.00,0.00,1.00}{##1}}}
\expandafter\def\csname PY@tok@ge\endcsname{\let\PY@it=\textit}
\expandafter\def\csname PY@tok@nt\endcsname{\let\PY@bf=\textbf\def\PY@tc##1{\textcolor[rgb]{0.00,0.50,0.00}{##1}}}
\expandafter\def\csname PY@tok@cm\endcsname{\let\PY@it=\textit\def\PY@tc##1{\textcolor[rgb]{0.25,0.50,0.50}{##1}}}
\expandafter\def\csname PY@tok@kd\endcsname{\let\PY@bf=\textbf\def\PY@tc##1{\textcolor[rgb]{0.00,0.50,0.00}{##1}}}
\expandafter\def\csname PY@tok@se\endcsname{\let\PY@bf=\textbf\def\PY@tc##1{\textcolor[rgb]{0.73,0.40,0.13}{##1}}}
\expandafter\def\csname PY@tok@m\endcsname{\def\PY@tc##1{\textcolor[rgb]{0.40,0.40,0.40}{##1}}}
\expandafter\def\csname PY@tok@gd\endcsname{\def\PY@tc##1{\textcolor[rgb]{0.63,0.00,0.00}{##1}}}
\expandafter\def\csname PY@tok@kc\endcsname{\let\PY@bf=\textbf\def\PY@tc##1{\textcolor[rgb]{0.00,0.50,0.00}{##1}}}
\expandafter\def\csname PY@tok@ch\endcsname{\let\PY@it=\textit\def\PY@tc##1{\textcolor[rgb]{0.25,0.50,0.50}{##1}}}
\expandafter\def\csname PY@tok@nv\endcsname{\def\PY@tc##1{\textcolor[rgb]{0.10,0.09,0.49}{##1}}}
\expandafter\def\csname PY@tok@nc\endcsname{\let\PY@bf=\textbf\def\PY@tc##1{\textcolor[rgb]{0.00,0.00,1.00}{##1}}}
\expandafter\def\csname PY@tok@sa\endcsname{\def\PY@tc##1{\textcolor[rgb]{0.73,0.13,0.13}{##1}}}
\expandafter\def\csname PY@tok@sh\endcsname{\def\PY@tc##1{\textcolor[rgb]{0.73,0.13,0.13}{##1}}}
\expandafter\def\csname PY@tok@w\endcsname{\def\PY@tc##1{\textcolor[rgb]{0.73,0.73,0.73}{##1}}}
\expandafter\def\csname PY@tok@cp\endcsname{\def\PY@tc##1{\textcolor[rgb]{0.74,0.48,0.00}{##1}}}
\expandafter\def\csname PY@tok@gt\endcsname{\def\PY@tc##1{\textcolor[rgb]{0.00,0.27,0.87}{##1}}}
\expandafter\def\csname PY@tok@kp\endcsname{\def\PY@tc##1{\textcolor[rgb]{0.00,0.50,0.00}{##1}}}
\expandafter\def\csname PY@tok@kn\endcsname{\let\PY@bf=\textbf\def\PY@tc##1{\textcolor[rgb]{0.00,0.50,0.00}{##1}}}
\expandafter\def\csname PY@tok@sb\endcsname{\def\PY@tc##1{\textcolor[rgb]{0.73,0.13,0.13}{##1}}}
\expandafter\def\csname PY@tok@mo\endcsname{\def\PY@tc##1{\textcolor[rgb]{0.40,0.40,0.40}{##1}}}
\expandafter\def\csname PY@tok@go\endcsname{\def\PY@tc##1{\textcolor[rgb]{0.53,0.53,0.53}{##1}}}
\expandafter\def\csname PY@tok@gs\endcsname{\let\PY@bf=\textbf}
\expandafter\def\csname PY@tok@nb\endcsname{\def\PY@tc##1{\textcolor[rgb]{0.00,0.50,0.00}{##1}}}
\expandafter\def\csname PY@tok@nf\endcsname{\def\PY@tc##1{\textcolor[rgb]{0.00,0.00,1.00}{##1}}}
\expandafter\def\csname PY@tok@gi\endcsname{\def\PY@tc##1{\textcolor[rgb]{0.00,0.63,0.00}{##1}}}
\expandafter\def\csname PY@tok@il\endcsname{\def\PY@tc##1{\textcolor[rgb]{0.40,0.40,0.40}{##1}}}
\expandafter\def\csname PY@tok@s2\endcsname{\def\PY@tc##1{\textcolor[rgb]{0.73,0.13,0.13}{##1}}}
\expandafter\def\csname PY@tok@k\endcsname{\let\PY@bf=\textbf\def\PY@tc##1{\textcolor[rgb]{0.00,0.50,0.00}{##1}}}
\expandafter\def\csname PY@tok@mi\endcsname{\def\PY@tc##1{\textcolor[rgb]{0.40,0.40,0.40}{##1}}}
\expandafter\def\csname PY@tok@sr\endcsname{\def\PY@tc##1{\textcolor[rgb]{0.73,0.40,0.53}{##1}}}
\expandafter\def\csname PY@tok@cs\endcsname{\let\PY@it=\textit\def\PY@tc##1{\textcolor[rgb]{0.25,0.50,0.50}{##1}}}
\expandafter\def\csname PY@tok@vc\endcsname{\def\PY@tc##1{\textcolor[rgb]{0.10,0.09,0.49}{##1}}}
\expandafter\def\csname PY@tok@nd\endcsname{\def\PY@tc##1{\textcolor[rgb]{0.67,0.13,1.00}{##1}}}
\expandafter\def\csname PY@tok@gh\endcsname{\let\PY@bf=\textbf\def\PY@tc##1{\textcolor[rgb]{0.00,0.00,0.50}{##1}}}
\expandafter\def\csname PY@tok@ni\endcsname{\let\PY@bf=\textbf\def\PY@tc##1{\textcolor[rgb]{0.60,0.60,0.60}{##1}}}
\expandafter\def\csname PY@tok@dl\endcsname{\def\PY@tc##1{\textcolor[rgb]{0.73,0.13,0.13}{##1}}}
\expandafter\def\csname PY@tok@s\endcsname{\def\PY@tc##1{\textcolor[rgb]{0.73,0.13,0.13}{##1}}}
\expandafter\def\csname PY@tok@gr\endcsname{\def\PY@tc##1{\textcolor[rgb]{1.00,0.00,0.00}{##1}}}
\expandafter\def\csname PY@tok@vi\endcsname{\def\PY@tc##1{\textcolor[rgb]{0.10,0.09,0.49}{##1}}}
\expandafter\def\csname PY@tok@c\endcsname{\let\PY@it=\textit\def\PY@tc##1{\textcolor[rgb]{0.25,0.50,0.50}{##1}}}
\expandafter\def\csname PY@tok@ow\endcsname{\let\PY@bf=\textbf\def\PY@tc##1{\textcolor[rgb]{0.67,0.13,1.00}{##1}}}
\expandafter\def\csname PY@tok@gu\endcsname{\let\PY@bf=\textbf\def\PY@tc##1{\textcolor[rgb]{0.50,0.00,0.50}{##1}}}
\expandafter\def\csname PY@tok@na\endcsname{\def\PY@tc##1{\textcolor[rgb]{0.49,0.56,0.16}{##1}}}
\expandafter\def\csname PY@tok@mb\endcsname{\def\PY@tc##1{\textcolor[rgb]{0.40,0.40,0.40}{##1}}}
\expandafter\def\csname PY@tok@s1\endcsname{\def\PY@tc##1{\textcolor[rgb]{0.73,0.13,0.13}{##1}}}
\expandafter\def\csname PY@tok@no\endcsname{\def\PY@tc##1{\textcolor[rgb]{0.53,0.00,0.00}{##1}}}
\expandafter\def\csname PY@tok@c1\endcsname{\let\PY@it=\textit\def\PY@tc##1{\textcolor[rgb]{0.25,0.50,0.50}{##1}}}
\expandafter\def\csname PY@tok@nl\endcsname{\def\PY@tc##1{\textcolor[rgb]{0.63,0.63,0.00}{##1}}}
\expandafter\def\csname PY@tok@mf\endcsname{\def\PY@tc##1{\textcolor[rgb]{0.40,0.40,0.40}{##1}}}
\expandafter\def\csname PY@tok@vg\endcsname{\def\PY@tc##1{\textcolor[rgb]{0.10,0.09,0.49}{##1}}}

\def\PYZbs{\char`\\}
\def\PYZus{\char`\_}
\def\PYZob{\char`\{}
\def\PYZcb{\char`\}}
\def\PYZca{\char`\^}
\def\PYZam{\char`\&}
\def\PYZlt{\char`\<}
\def\PYZgt{\char`\>}
\def\PYZsh{\char`\#}
\def\PYZpc{\char`\%}
\def\PYZdl{\char`\$}
\def\PYZhy{\char`\-}
\def\PYZsq{\char`\'}
\def\PYZdq{\char`\"}
\def\PYZti{\char`\~}
% for compatibility with earlier versions
\def\PYZat{@}
\def\PYZlb{[}
\def\PYZrb{]}
\makeatother


    % Exact colors from NB
    \definecolor{incolor}{rgb}{0.0, 0.0, 0.5}
    \definecolor{outcolor}{rgb}{0.545, 0.0, 0.0}



    
    % Prevent overflowing lines due to hard-to-break entities
    \sloppy 
    % Setup hyperref package
    \hypersetup{
      breaklinks=true,  % so long urls are correctly broken across lines
      colorlinks=true,
      urlcolor=urlcolor,
      linkcolor=linkcolor,
      citecolor=citecolor,
      }
    % Slightly bigger margins than the latex defaults
    
    \geometry{verbose,tmargin=1in,bmargin=1in,lmargin=1in,rmargin=1in}
    
    

    \begin{document}
    
    
    \maketitle
    
    

    
    \section{BP算法}\label{bpux7b97ux6cd5}

\begin{verbatim}
BP算法详细描述,参考韩力群的《人工神经网络教程》P58 - P64
\end{verbatim}

\paragraph{基于BP算法的多层感知器模型}\label{ux57faux4e8ebpux7b97ux6cd5ux7684ux591aux5c42ux611fux77e5ux5668ux6a21ux578b}

三层BP网络

输入向量:X=\((x_{1},x_{2},...,x_{i},...,x_{n})^T\)
,\(x_{0}=-1\)是为隐层神经元引入阈值而设置.

隐层输出向量:Y=\((y_{1},y_{2},...,y_{j},...,y_{m})^T\)
,\(y_{0}=-1\)是为输出层神经元引入阈值而设置.

输出层输出向量:O=\((o_{1},o_{2},...,o_{k},...,o_{l})^T\).

输入层到隐层之间的权值矩阵用V表示,V=\((V_{1},V_{2},...,V_{j},...,V_{m})\).其中列向量\(V_{j}\)为隐层第j个神经元对应的权向量.

隐层到输出层之间的权值矩阵用W表示,V=\((W_{1},W_{2},...,W_{k},...,W_{l})\).其中列向量\(W_{k}\)为输出层第k个神经元对应的权向量.

各层之间的数学关系:

对于输出层,有

\$o\_\{k\}=f(net\_\{k\}) \qquad\qquad  k=1,2,...,l \$ \(\qquad\)(3.10)

\$net\_\{k\}=\sum\emph{\{j=0\}\^{}mw}\{jk\}y\_\{j\}\qquad  k=1,2,...,l
\$ \(\qquad\) (3.11)

对于隐层,有

\$y\_\{j\}=f(net\_\{j\}) \qquad\qquad j=1,2,...,m \$ \(\qquad\) (3.12)

\(net_{j}=\sum_{i=0}^nv_{ij}x_{i} \qquad j=1,2,...,m\) \(\qquad\) (3.13)

变换函数为Sigmoid函数

\(f(x)=\frac{1}{1+e^{-x}}\) \(\qquad\) (3.14)

\(f'(x)=f(x)(1-f(x))\) \(\qquad\) (3.15)

\paragraph{BP学习算法}\label{bpux5b66ux4e60ux7b97ux6cd5}

\begin{verbatim}
下面以三层感知器为例介绍BP学习算法
\end{verbatim}

\begin{itemize}
\tightlist
\item
  网络误差与权值调整
\end{itemize}

当网络输出与期望输出不等时,存在输出误差E,定义如下:

\(E=\frac{1}{2}(d-o)^2=\frac{1}{2}\sum_{k=1}^l(d_{k}-o_{k})^2\)
\(\qquad\) (3.16)

将以上定义展开至隐层

\$E=\frac{1}{2}(d-o)\textsuperscript{2=\frac{1}{2}\sum\emph{\{k=1\}\textsuperscript{l(d\_\{k\}-o\_\{k\})}2
\textbackslash{}
=\frac{1}{2}\sum}\{k=1\}}l{[}d\_\{k\}-f(net\_\{k\}){]}\^{}2
\textbackslash{}
=\frac{1}{2}\sum\_\{k=1\}\textsuperscript{l{[}d\_\{k\}-f(\sum\emph{\{j=0\}\^{}mw}\{jk\}y\_\{j\}){]}}2
\textbackslash{} \$ \(\qquad\) (3.17)

进一步展开至输入层

\$E=\frac{1}{2}(d-o)\textsuperscript{2=\frac{1}{2}\sum\emph{\{k=1\}\textsuperscript{l(d\_\{k\}-o\_\{k\})}2
\textbackslash{}
=\frac{1}{2}\sum}\{k=1\}}l{[}d\_\{k\}-f(net\_\{k\}){]}\^{}2
\textbackslash{}
=\frac{1}{2}\sum\emph{\{k=1\}\textsuperscript{l{[}d\_\{k\}-f(\sum\emph{\{j=0\}\^{}mw}\{jk\}y\_\{j\}){]}}2
\textbackslash{}
=\frac{1}{2}\sum}\{k=1\}\textsuperscript{l{[}d\_\{k\}-f(\sum\emph{\{j=0\}\^{}mw}\{jk\}f(net\_\{j\})){]}}2
\textbackslash{}
=\frac{1}{2}\sum\_\{k=1\}\textsuperscript{l{[}d\_\{k\}-f(\sum\emph{\{j=0\}\^{}mw}\{jk\}f(\sum\emph{\{i=0\}\^{}nv}\{ij\}x\_\{i\})){]}}2
\textbackslash{} \$ \$\qquad \$(3.18)

由上式可以看出,网络输入误差是各层权值\(v_{ij},w_{jk}\)的函数,因此调整权值可以改变误差E.

显然,调整权值的原则是使误差不断减少,因此应使权值的调整量与误差的梯度下降成正比,即

\(\Delta w_{jk}=-\eta \frac{\partial E}{\partial w_{jk}} \qquad j=0,1,2,...,m \qquad k=1,2,...,l\)
\(\qquad\) (3.19a)

\(\Delta v_{ij}=-\eta \frac{\partial E}{\partial v_{jk}} \qquad i=0,1,2,...,n \qquad j=1,2,...,m\)
\(\qquad\) (3.19b)

式中符号表示梯度下降,常数\(\eta\)表示学习率 {[}0,1{]}

\begin{itemize}
\tightlist
\item
  BP算法推导
\end{itemize}

\(\Delta w_{jk}=-\eta \frac{\partial E}{\partial w_{jk}} =-\eta \frac{\partial E}{\partial net_{k}} \frac{\partial net_{k}}{\partial w_{jk}}\)
\(\qquad\) (3.20a)

\(\Delta v_{ij}=-\eta \frac{\partial E}{\partial v_{ij}} =-\eta \frac{\partial E}{\partial net_{j}} \frac{\partial net_{j}}{\partial v_{ij}}\)
\(\qquad\) (3.20b)

定义误差信号,令

\(\delta ^o_{k}=-\frac{\partial E}{\partial net_{k}}\) \(\qquad\)
(3.21a)

\(\delta ^y_{j}=-\frac{\partial E}{\partial net_{j}}\) \(\qquad\)
(3.21b)

则

\(\Delta w_{jk}=\eta \delta^o_{k} y_{j}\) \(\qquad\) (3.22a)

\(\Delta v_{ij}=\eta \delta^y_{j} x_{i}\) \(\qquad\) (3.22b)

展开\(\delta\)

\(\delta ^o_{k}=-\frac{\partial E}{\partial net_{k}} =-\frac{\partial E}{\partial o_{k}} \frac{\partial o_{k}}{\partial net_{k}} =-\frac{\partial E}{\partial o_{k}} f'(net_{k})\)
\(\qquad\) (3.23a)

\(\delta ^y_{j}=-\frac{\partial E}{\partial net_{j}} =-\frac{\partial E}{\partial y_{j}} \frac{\partial y_{j}}{\partial net_{j}} =-\frac{\partial E}{\partial y_{j}} f'(net_{j})\)
\(\qquad\) (3.23b)

\(\frac{\partial E}{\partial o_{k}}=-(d_{k}-o_{k})\) \(\qquad\) (3.24a)

\(\frac{\partial E}{\partial y_{j}} =-\sum_{k=1}^l(d_{k}-o_{k})f'(net_{k})w_{jk}\)
\(\qquad\) (3.24b)

\$\delta \^{}o\_\{k\}=-\frac{\partial E}{\partial net_{k}}
=(d\_\{k\}-o\_\{k\})o\_\{k\}(1-o\_\{k\}) \$ \(\qquad\) (3.25a)

\$\delta \^{}y\_\{j\}=-\frac{\partial E}{\partial net_{j}}
={[}\sum\emph{\{k=1\}\^{}l(d}\{k\}-o\_\{k\})f'(net\_\{k\})w\_\{jk\}{]}f'(net\_\{j\})
=(\sum\emph{\{k=1\}\textsuperscript{l\delta }o}\{k\}w\_\{jk\})y\_\{j\}(1-y\_\{j\})
\$ \(\qquad\) (3.25b)

权重调整公式

\(\Delta w_{jk}=\eta \delta^o_{k} y_{j}=\eta (d_{k}-o_{k})o_{k}(1-o_{k}) y_{j}\)
\(\qquad\) (3.26a)

\(\Delta v_{ij}=\eta \delta^y_{j} x_{i}=\eta (\sum_{k=1}^l\delta ^o_{k}w_{jk})y_{j}(1-y_{j}) x_{i}\)
\(\qquad\) (3.26b)

\paragraph{BP算法的程序实现}\label{bpux7b97ux6cd5ux7684ux7a0bux5e8fux5b9eux73b0}

    \begin{Verbatim}[commandchars=\\\{\}]
{\color{incolor}In [{\color{incolor} }]:} \PY{k+kn}{import} \PY{n+nn}{numpy} \PY{k}{as} \PY{n+nn}{np}
        \PY{c+c1}{\PYZsh{}代码中使用变量/函数解析}
        \PY{c+c1}{\PYZsh{}P  \PYZsh{}样本数}
        \PY{c+c1}{\PYZsh{}n  \PYZsh{}输入向量长度}
        \PY{c+c1}{\PYZsh{}m  \PYZsh{}隐藏层神经元个数}
        \PY{c+c1}{\PYZsh{}l  \PYZsh{}输出向量长度}
        
        \PY{c+c1}{\PYZsh{}x  \PYZsh{}输入矩阵       P x n}
        \PY{c+c1}{\PYZsh{}y  \PYZsh{}隐层输出矩阵    P x m}
        \PY{c+c1}{\PYZsh{}o  \PYZsh{}输出层输出矩阵  P x l}
        \PY{c+c1}{\PYZsh{}d  \PYZsh{}期望输出值矩阵  P x l}
        \PY{c+c1}{\PYZsh{}f  \PYZsh{}变换函数}
        
        \PY{c+c1}{\PYZsh{}v  \PYZsh{}输入到隐层之间的权值矩阵  m x n}
        \PY{c+c1}{\PYZsh{}w  \PYZsh{}隐层到输出之间的权值矩阵  l x m}
        
        \PY{c+c1}{\PYZsh{}err\PYZus{}total \PYZsh{}计算总误差}
        
        \PY{n}{P}\PY{o}{=}\PY{l+m+mi}{1}
        \PY{n}{n}\PY{o}{=}\PY{l+m+mi}{2}
        \PY{n}{m}\PY{o}{=}\PY{l+m+mi}{2}
        \PY{n}{l}\PY{o}{=}\PY{l+m+mi}{2}
        \PY{n}{lamda}\PY{o}{=}\PY{l+m+mf}{1.0} \PY{c+c1}{\PYZsh{}学习率}
        \PY{n}{epochs}\PY{o}{=}\PY{l+m+mi}{20}
        
        \PY{c+c1}{\PYZsh{}构造训练样本}
        \PY{c+c1}{\PYZsh{}x=np.array([[0.2,1.2]],dtype=np.float32)         \PYZsh{}输入向量}
        \PY{c+c1}{\PYZsh{}d=np.array([[0.4,0.3,0.2,0.1]],dtype=np.float32) \PYZsh{}期望值}
        \PY{n}{x}\PY{o}{=}\PY{n}{np}\PY{o}{.}\PY{n}{random}\PY{o}{.}\PY{n}{rand}\PY{p}{(}\PY{n}{P}\PY{p}{,}\PY{n}{n}\PY{p}{)} \PY{c+c1}{\PYZsh{}输入矩阵 ,[P x n]}
        \PY{n}{d}\PY{o}{=}\PY{n}{np}\PY{o}{.}\PY{n}{random}\PY{o}{.}\PY{n}{rand}\PY{p}{(}\PY{n}{P}\PY{p}{,}\PY{n}{l}\PY{p}{)} \PY{c+c1}{\PYZsh{}期望矩阵 ,[P x l]}
        \PY{n}{P}\PY{o}{=}\PY{n}{d}\PY{o}{.}\PY{n}{shape}\PY{p}{[}\PY{l+m+mi}{0}\PY{p}{]}          \PY{c+c1}{\PYZsh{}样本数}
        \PY{n}{n}\PY{o}{=}\PY{n}{x}\PY{o}{.}\PY{n}{shape}\PY{p}{[}\PY{l+m+mi}{1}\PY{p}{]}          \PY{c+c1}{\PYZsh{}输入向量长度}
        \PY{n}{l}\PY{o}{=}\PY{n}{d}\PY{o}{.}\PY{n}{shape}\PY{p}{[}\PY{l+m+mi}{1}\PY{p}{]}          \PY{c+c1}{\PYZsh{}输出向量长度}
        
        \PY{c+c1}{\PYZsh{}初始化V,W}
        \PY{n}{v}\PY{o}{=}\PY{n}{np}\PY{o}{.}\PY{n}{random}\PY{o}{.}\PY{n}{rand}\PY{p}{(}\PY{n}{m}\PY{p}{,}\PY{n}{n}\PY{p}{)}  \PY{c+c1}{\PYZsh{} m x n}
        \PY{n}{w}\PY{o}{=}\PY{n}{np}\PY{o}{.}\PY{n}{random}\PY{o}{.}\PY{n}{rand}\PY{p}{(}\PY{n}{l}\PY{p}{,}\PY{n}{m}\PY{p}{)}  \PY{c+c1}{\PYZsh{} l x m}
        
        \PY{n+nb}{print}\PY{p}{(}\PY{l+s+s1}{\PYZsq{}}\PY{l+s+s1}{x.shape:}\PY{l+s+si}{\PYZpc{}s}\PY{l+s+s1}{,d.shape:}\PY{l+s+si}{\PYZpc{}s}\PY{l+s+s1}{,v.shape:}\PY{l+s+si}{\PYZpc{}s}\PY{l+s+s1}{,w.shape:}\PY{l+s+si}{\PYZpc{}s}\PY{l+s+s1}{\PYZsq{}}\PY{o}{\PYZpc{}}\PY{p}{(}\PY{n}{x}\PY{o}{.}\PY{n}{shape}\PY{p}{,}\PY{n}{d}\PY{o}{.}\PY{n}{shape}\PY{p}{,}\PY{n}{v}\PY{o}{.}\PY{n}{shape}\PY{p}{,}\PY{n}{w}\PY{o}{.}\PY{n}{shape}\PY{p}{)}\PY{p}{)}
        \PY{n+nb}{print}\PY{p}{(}\PY{l+s+s1}{\PYZsq{}}\PY{l+s+s1}{P:}\PY{l+s+si}{\PYZpc{}s}\PY{l+s+s1}{,n:}\PY{l+s+si}{\PYZpc{}s}\PY{l+s+s1}{,m:}\PY{l+s+si}{\PYZpc{}s}\PY{l+s+s1}{,l:}\PY{l+s+si}{\PYZpc{}s}\PY{l+s+s1}{\PYZsq{}}\PY{o}{\PYZpc{}}\PY{p}{(}\PY{n}{P}\PY{p}{,}\PY{n}{n}\PY{p}{,}\PY{n}{m}\PY{p}{,}\PY{n}{l}\PY{p}{)}\PY{p}{)}
        
        \PY{c+c1}{\PYZsh{}sigmoid function}
        \PY{k}{def} \PY{n+nf}{f}\PY{p}{(}\PY{n}{x}\PY{p}{,} \PY{n}{deriv} \PY{o}{=} \PY{k+kc}{False}\PY{p}{)}\PY{p}{:}
            \PY{k}{if} \PY{p}{(}\PY{n}{deriv} \PY{o}{==} \PY{k+kc}{True}\PY{p}{)}\PY{p}{:}
                \PY{k}{return} \PY{n}{x} \PY{o}{*} \PY{p}{(}\PY{l+m+mi}{1} \PY{o}{\PYZhy{}} \PY{n}{x}\PY{p}{)}       \PY{c+c1}{\PYZsh{}如果deriv为true,求导数,此时的x为节点输出值,即f(x)}
            \PY{k}{return} \PY{l+m+mi}{1} \PY{o}{/} \PY{p}{(}\PY{l+m+mi}{1} \PY{o}{+} \PY{n}{np}\PY{o}{.}\PY{n}{exp}\PY{p}{(}\PY{o}{\PYZhy{}}\PY{n}{x}\PY{p}{)}\PY{p}{)}  \PY{c+c1}{\PYZsh{}exp()是以e为底的指数函数}
        
        \PY{c+c1}{\PYZsh{}计算总误差}
        \PY{k}{def} \PY{n+nf}{err\PYZus{}total}\PY{p}{(}\PY{n}{d}\PY{p}{,}\PY{n}{o}\PY{p}{)}\PY{p}{:}
            \PY{c+c1}{\PYZsh{}d=np.array([[1,2,3],[4,7,6]],dtype=np.float32)}
            \PY{c+c1}{\PYZsh{}o=np.array([[1,3,3],[4,5,6]],dtype=np.float32)}
            \PY{c+c1}{\PYZsh{}print(err\PYZus{}total(d,o))}
            \PY{k}{return} \PY{n}{np}\PY{o}{.}\PY{n}{sqrt}\PY{p}{(}\PY{n}{np}\PY{o}{.}\PY{n}{sum}\PY{p}{(}\PY{p}{(}\PY{n}{d}\PY{o}{\PYZhy{}}\PY{n}{o}\PY{p}{)}\PY{o}{*}\PY{o}{*}\PY{l+m+mi}{2}\PY{p}{)}\PY{p}{)}
        
        \PY{k}{for} \PY{n}{i} \PY{o+ow}{in} \PY{n+nb}{range}\PY{p}{(}\PY{n}{epochs}\PY{p}{)}\PY{p}{:}
            \PY{k}{for} \PY{n}{xx}\PY{p}{,}\PY{n}{dd} \PY{o+ow}{in} \PY{n+nb}{zip}\PY{p}{(}\PY{n}{x}\PY{p}{,}\PY{n}{d}\PY{p}{)}\PY{p}{:}
                \PY{n+nb}{print}\PY{p}{(}\PY{l+s+s1}{\PYZsq{}}\PY{l+s+s1}{\PYZhy{}\PYZhy{}\PYZhy{}\PYZhy{}\PYZhy{}\PYZhy{}\PYZhy{}\PYZhy{}\PYZhy{}\PYZhy{}}\PY{l+s+s1}{\PYZsq{}}\PY{p}{)}
                \PY{c+c1}{\PYZsh{}print(\PYZsq{}xx.shape:\PYZsq{},xx.shape) \PYZsh{}(n,)}
                \PY{c+c1}{\PYZsh{}print(\PYZsq{}dd.shape:\PYZsq{},dd.shape) \PYZsh{}(l,)}
                
                \PY{c+c1}{\PYZsh{}正向传播计算}
                \PY{n}{yy}\PY{o}{=}\PY{n}{f}\PY{p}{(}\PY{n}{np}\PY{o}{.}\PY{n}{dot}\PY{p}{(}\PY{n}{v}\PY{p}{,}\PY{n}{xx}\PY{p}{)}\PY{p}{)}  \PY{c+c1}{\PYZsh{}[m x n] * [n x 1] =\PYZgt{} [m x 1] \PYZsh{}m个隐层节点}
                \PY{n}{oo}\PY{o}{=}\PY{n}{f}\PY{p}{(}\PY{n}{np}\PY{o}{.}\PY{n}{dot}\PY{p}{(}\PY{n}{w}\PY{p}{,}\PY{n}{yy}\PY{p}{)}\PY{p}{)}  \PY{c+c1}{\PYZsh{}[l x m] *[m x 1]  =\PYZgt{} [l x 1] \PYZsh{}l个输出节点}
                \PY{c+c1}{\PYZsh{}print(\PYZsq{}yy.shape:\PYZsq{},yy.shape) \PYZsh{}(m,)}
                \PY{c+c1}{\PYZsh{}print(\PYZsq{}oo.shape:\PYZsq{},oo.shape) \PYZsh{}(l,)}
                \PY{c+c1}{\PYZsh{}print(\PYZsq{}v.shape:\PYZsq{},v.shape)   \PYZsh{}(n,m)}
                \PY{c+c1}{\PYZsh{}print(\PYZsq{}w.shape:\PYZsq{},w.shape)   \PYZsh{}(l,m)}
                
                \PY{c+c1}{\PYZsh{}反向传播计算}
                \PY{c+c1}{\PYZsh{}公式(3.25a)}
                \PY{c+c1}{\PYZsh{}delta\PYZus{}o[k]=(dd[k]\PYZhy{}oo[k])*(1\PYZhy{}oo[k])*oo[k]}
                \PY{n}{delta\PYZus{}o}\PY{o}{=}\PY{p}{(}\PY{n}{dd}\PY{o}{\PYZhy{}}\PY{n}{oo}\PY{p}{)}\PY{o}{*}\PY{p}{(}\PY{l+m+mi}{1}\PY{o}{\PYZhy{}}\PY{n}{oo}\PY{p}{)}\PY{o}{*}\PY{n}{oo}  \PY{c+c1}{\PYZsh{}elementwise,计算每个输出节点的梯度值}
                \PY{c+c1}{\PYZsh{}公式 (3.25b)}
                \PY{c+c1}{\PYZsh{}np.dot(delta\PYZus{}o,w) =\PYZgt{} 隐层节点对应每个输出节点的梯度值的加权和}
                \PY{n}{delta\PYZus{}y}\PY{o}{=}\PY{n}{np}\PY{o}{.}\PY{n}{dot}\PY{p}{(}\PY{n}{delta\PYZus{}o}\PY{p}{,}\PY{n}{w}\PY{p}{)}\PY{o}{*}\PY{p}{(}\PY{l+m+mi}{1}\PY{o}{\PYZhy{}}\PY{n}{yy}\PY{p}{)}\PY{o}{*}\PY{n}{yy}   \PY{c+c1}{\PYZsh{}np.dot(delta\PYZus{}o,w).shape=\PYZgt{}(m,)}
                \PY{c+c1}{\PYZsh{}print(\PYZsq{}delta\PYZus{}o.shape:\PYZsq{},delta\PYZus{}o.shape) \PYZsh{}(l,)}
                \PY{c+c1}{\PYZsh{}print(\PYZsq{}delta\PYZus{}y.shape:\PYZsq{},delta\PYZus{}y.shape) \PYZsh{}(m,)}
        
                \PY{c+c1}{\PYZsh{}更新权重}
                \PY{n}{w}\PY{o}{+}\PY{o}{=}\PY{n}{lamda}\PY{o}{*}\PY{n}{np}\PY{o}{.}\PY{n}{dot}\PY{p}{(}\PY{n}{delta\PYZus{}o}\PY{o}{.}\PY{n}{reshape}\PY{p}{(}\PY{n}{l}\PY{p}{,}\PY{l+m+mi}{1}\PY{p}{)}\PY{p}{,}\PY{n}{yy}\PY{o}{.}\PY{n}{reshape}\PY{p}{(}\PY{l+m+mi}{1}\PY{p}{,}\PY{n}{m}\PY{p}{)}\PY{p}{)}  \PY{c+c1}{\PYZsh{}公式 (3.26a),\PYZca{}w[j,k]+=lambda*o[k]*y[j]}
                \PY{n}{v}\PY{o}{+}\PY{o}{=}\PY{n}{lamda}\PY{o}{*}\PY{n}{np}\PY{o}{.}\PY{n}{dot}\PY{p}{(}\PY{n}{delta\PYZus{}y}\PY{o}{.}\PY{n}{reshape}\PY{p}{(}\PY{n}{m}\PY{p}{,}\PY{l+m+mi}{1}\PY{p}{)}\PY{p}{,}\PY{n}{xx}\PY{o}{.}\PY{n}{reshape}\PY{p}{(}\PY{l+m+mi}{1}\PY{p}{,}\PY{n}{n}\PY{p}{)}\PY{p}{)}  \PY{c+c1}{\PYZsh{}公式 (3.26b),\PYZca{}v[i,j]+=lambda*y[j]*x[i]}
                
                \PY{c+c1}{\PYZsh{}\PYZhy{}\PYZhy{}\PYZhy{}\PYZhy{}\PYZhy{}\PYZhy{}\PYZhy{}\PYZhy{}\PYZhy{}\PYZhy{}\PYZhy{}\PYZhy{}\PYZhy{}\PYZhy{}\PYZhy{}\PYZhy{}\PYZhy{}\PYZhy{}}
                \PY{n+nb}{print}\PY{p}{(}\PY{n}{w}\PY{o}{.}\PY{n}{reshape}\PY{p}{(}\PY{o}{\PYZhy{}}\PY{l+m+mi}{1}\PY{p}{)}\PY{p}{)}
                \PY{n+nb}{print}\PY{p}{(}\PY{n}{v}\PY{o}{.}\PY{n}{reshape}\PY{p}{(}\PY{o}{\PYZhy{}}\PY{l+m+mi}{1}\PY{p}{)}\PY{p}{)}
                \PY{c+c1}{\PYZsh{}print(w)}
                \PY{c+c1}{\PYZsh{}print(v)}
\end{Verbatim}


    % Add a bibliography block to the postdoc
    
    
    
    \end{document}
